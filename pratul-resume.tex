%%%%%%
% Pratul Kalia - curriculum vitae
% Edited using emacs and texlive
% Uses LaTeX and moderncv
%%%%%%

\documentclass[11pt,a4paper]{moderncv}
\moderncvtheme[grey]{classic}

% character encoding
\usepackage[utf8]{inputenc}

% adjust the page margins
\usepackage[scale=0.8]{geometry}
%\setlength{\hintscolumnwidth}{3cm}
\AtBeginDocument{\setlength{\maketitlenamewidth}{10cm}}
\AtBeginDocument{\recomputelengths}                     % required when changes are made to page layout lengths

% personal data
\firstname{Pratul}
\familyname{Kalia}
\title{Curriculum Vitae}
\address{House Number 622, Sector 19}{Faridabad, India}
\phone{+91-9953830040}
\email{pratul@pratul.in}
% \extrainfo{additional information (optional)} % optional, remove the line if not wanted
% \photo[64pt]{picture}                         % '64pt' is the height the picture must be resized to and 'picture' is the name of the picture file; optional, remove the line if not wanted
%\quote{Some quote (optional)}                 % optional, remove the line if not wanted

%\nopagenumbers{}

% --------------------------------------
% CONTENT
% --------------------------------------

\begin{document}
\maketitle{}

\section{Education}
\cvlistitem{Pursuing a Bachelors degree in \textit{Computer Science Engineering} from Maharshi Dayanand University, Rohtak.}

\section{Technical skills}
\cvcomputer
{\textbf{Operating Systems}}{Linux, Mac OS X, Windows}
{\textbf{Programming Languages}}{C, PHP, HTML/CSS, JavaScript, Objective-C, Python}
\cvcomputer
{\textbf{Development Tools}}{emacs, gdb, git, cvs/svn, zsh/bash, \\Xcode, TextMate}
{\textbf{Frameworks}}{Drupal, WordPress, Cocoa (OS X)}
\cvcomputer
{\textbf{Software Packages}}{MySQL, SQLite, Adobe Photoshop, \LaTeX{}}
{}{}

% --------------------------------------
% WORK EX
% --------------------------------------

\section{Work Experience and Positions}

\cventry{May 2009 -- Aug 2009}{Google Summer of Code 2009}{Student Contractor}{Google}{}{Developed an advanced Ajax based voting framework module for the Drupal project (http://drupal.org) called Vote Up/Down (http://drupal.org/project/vote\_up\_down)}

\subsection{F/OSS Contributions}

\cventry{Sept 2009 -- present}{Goonj}{Founder Developer}{}{http://goonj.github.com}{Working on an advanced open source music player/manager for Mac OS X.}

\cventry{Jun 2008 -- present}
{Drupal}{Developer and Contributor}{}{http://drupal.org}
{Member of the Drupal.org Site Administrators team.
\\Member of the Drupal.org Documentation team.
\\Fixing and patching certain issues of the Core repository.
\\Maintenance of 2 modules in the Contributions repository (\href{htp://drupal.org/project/cumulus}{Cumulus} and \href{http://drupal.org/project/vote_up_down}{Vote Up/Down}).}

\cventry{Feb 2007 -- present}
{Ubuntu India}{Forum Administrator}{}{http://forum.ubuntu-in.info}
{Setup, hosting and administration of the forum.}

%%%%%%%%%%%
\subsection{Voluntary Work}
%%%%%%%%%%%

\cventry{Mar 2009}{Genesis 2009}{Developer}{College cultural festival}{http://genesis-bsaitm.com}
{Developed and partly designed the portal in Drupal 6.}

\cventry{Aug 2008}{OSScamp.in}{Developer}{One of India's largest unconferences}{http://osscamp.in}
{Implemented the website and required features using Drupal 6.}


% --------------------------------------
% PROJECTS
% --------------------------------------

\pagebreak
\section{Projects}

\cvlistitem
{\textbf{Vote Up/Down} -- {\small http://drupal.org/project/vote\_up\_down}
\\A Drupal module, Allows Ajax based voting on different Drupal objects like nodes, comments, taxonomy terms etc. Supports custom widget themes. Supports adding more voteable objects using code.\\
}

\cvlistitem
{\textbf{Cumulus} -- {\small http://drupal.org/project/cumulus}
\\A Drupal module, that was ported from the WP-Cumulus extension for WordPress. It allows you to display a website's tags or taxonomies using Flash, that rotates them beautifully in 3D. Works like a regular tag cloud, but is more visually exciting.\\
}

\cvlistitem
{\textbf{Almost Monokai} -- {\small http://github.com/lut4rp/almost-monokai}
\\A color theme for GNU Emacs and Apple Xcode, inspired from the Monokai TextMate theme.\\
}

%%%%%
\subsection{College and School}
%%%%%

\cvlistitem
{\textbf{Goonj} -- {\small http://goonj.github.com}
\\An advanced open source music player/manager for Mac OS X.
\\Submitted as the final year (7th \& 8th semester) project.\\
}

\cvlistitem
{\textbf{Shortkut}
\\A small ASP.NET web application written using the ASP.NET MVC framework and C\#. Converts long URLs into short ones, that the user can easily remember. Short URLs get HTTP redirected to longer ones automatically.
\\Submitted as the 6th semester project.\\
}

\cvlistitem
{\textbf{Poonji}
\\A basic text editor written in Python using the Tcl/Tk GUI library.
\\Submitted as the 4th semester project.\\
}

\cvlistitem
{\textbf{Nursing Home Automation System}
\\An automation and data management system for nursing homes, written in Visual Basic 6 and an Oracle 9i database backend.
\\Submitted as the school (class 12th) project.\\
}

\pagebreak
% --------------------------------------
% TALKS ETC.
% --------------------------------------

\section{Talks and Travels}

\cventry{Feb 2010}{Starting Your Own Open Source Project}{FOSSkriti '10, IIT Kanpur}{}{}
{A talk on how and when to start your own open source project, looking for alternatives, version control, issue tracking etc.}
\cventry{Jan 2010}{Version Control with Git}{OSScamp Delhi}{}{}
{Demo on the power and use of revision control, using the Git distributed version control system.}
\cventry{Mar 2009}{Delhi Drupal Meetup}{with LUG -- IIT Delhi}{}{}
{Helped organize the event, gave a demo of module development.}
\cventry{Mar 2009}{The Power of GNU Emacs}{OSScamp Delhi}{(with Prateek Saxena)}{}
{An introductory talk and demo about the GNU emacs editor, and its continuing relevance.}
\cventry{Feb 2009}{Drupal 7 Hackfest}{FOSSkriti '09, IIT Kanpur}{(with Gurpartap Singh)}{}
{A demo to Drupal module development, accompanied with a hands-on workshop on setup, code \& the latest Drupal version, 7.}
\cventry{Sept 2008}{Demoing Drupal Module Development}{OSScamp Delhi}{}{}
{A demo on how to develop your own Drupal module.}
\cventry{Jun 2008}{Comparing Linux Distributions}{OSScamp Delhi}{}{}
{A discussion comparing the features and specific uses of different Linux distributions, like Ubuntu, Debian, Gentoo, Fedora, Damn Small Linux and Red Hat.}
\cventry{Sept 2007}{Drupal CMS: half an hour super guide for newbies}{OSScamp Delhi}{}{}
{A basic introduction to the power and flexibility of Drupal, not including development.\\}

\cventry{Sept 2009}{DrupalCon Paris}{on scholarship}{Cité Universitaire, Paris}{}
{Was the only developer from India to get a full scholarship (travel + accommodation) to attend and interact with other developers at \href{http://paris2009.drupalcon.org}{DrupalCon Paris}, the official bi-annual Drupal community conference.}

% --------------------------------------
% PUBLICATIONS
% --------------------------------------

\section{Publications}
\cventry{Dec 2008}{The Intrepid Ibex Awaits Your Command}{Linux For You magazine}{}{}
{An article on the technical review of Ubuntu 8.10, code named ``Intrepid Ibex''}
\cventry{Nov 2008}{Unconferences For The Win}{Linux For You magazine}{}{}
{An article on the concept of unconferences and their growing popularity.}

% --------------------------------------
% INTERESTS HOBBIES
% --------------------------------------

\section{Interests and Hobbies}
\cvlistitem{An avid reader on technology, fiction and non-fiction books and websites, some of my favorite authors being Douglas Adams \& J.R.R. Tolkein.}

\cvlistitem{Hobbyist photographer -- http://flickr.com/pratulkalia}

\cvlistitem{I play the drums and have participated in various inter-college events. I listen to a wide variety of music, ranging from Indian classical to metal.}

\cvlistitem{I believe open source is a significantly better software development model, compared to the traditional ones. I've helped organize OSScamp Delhi a lot of times. I helped initiate and organize the Delhi Drupal Meetup, held in March 2009.}

\end{document}